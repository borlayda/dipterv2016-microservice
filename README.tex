\documentclass[]{article}
\usepackage{lmodern}
\usepackage{amssymb,amsmath}
\usepackage{ifxetex,ifluatex}
\usepackage{fixltx2e} % provides \textsubscript
\ifnum 0\ifxetex 1\fi\ifluatex 1\fi=0 % if pdftex
  \usepackage[T1]{fontenc}
  \usepackage[utf8]{inputenc}
\else % if luatex or xelatex
  \ifxetex
    \usepackage{mathspec}
  \else
    \usepackage{fontspec}
  \fi
  \defaultfontfeatures{Ligatures=TeX,Scale=MatchLowercase}
\fi
% use upquote if available, for straight quotes in verbatim environments
\IfFileExists{upquote.sty}{\usepackage{upquote}}{}
% use microtype if available
\IfFileExists{microtype.sty}{%
\usepackage{microtype}
\UseMicrotypeSet[protrusion]{basicmath} % disable protrusion for tt fonts
}{}
\usepackage{hyperref}
\hypersetup{unicode=true,
            pdfborder={0 0 0},
            breaklinks=true}
\urlstyle{same}  % don't use monospace font for urls
\IfFileExists{parskip.sty}{%
\usepackage{parskip}
}{% else
\setlength{\parindent}{0pt}
\setlength{\parskip}{6pt plus 2pt minus 1pt}
}
\setlength{\emergencystretch}{3em}  % prevent overfull lines
\providecommand{\tightlist}{%
  \setlength{\itemsep}{0pt}\setlength{\parskip}{0pt}}
\setcounter{secnumdepth}{0}
% Redefines (sub)paragraphs to behave more like sections
\ifx\paragraph\undefined\else
\let\oldparagraph\paragraph
\renewcommand{\paragraph}[1]{\oldparagraph{#1}\mbox{}}
\fi
\ifx\subparagraph\undefined\else
\let\oldsubparagraph\subparagraph
\renewcommand{\subparagraph}[1]{\oldsubparagraph{#1}\mbox{}}
\fi

\date{}

\begin{document}

\section{Mikroszolgáltatásokra épülő architektúra fejlesztésének és
tesztelésének
támogatása}\label{mikroszolguxe1ltatuxe1sokra-uxe9puxfclux151-architektuxfara-fejlesztuxe9suxe9nek-uxe9s-teszteluxe9suxe9nek-tuxe1mogatuxe1sa}

\href{https://github.com/borlayda/dipterv2016-microservice/wiki}{Documentation}

\subsection{Alkalmazás leírás:}\label{alkalmazuxe1s-leuxedruxe1s}

Az alkalmazás, amin ketesztül a mikró szolgáltatások működését
bemutatom, egy könyvesbolt webárúháza lesz, ami rendelkezik egy webes
felülettel. A felhasználó be tud jelentkezni a felületre, és tud
könyveket vásárolni magának.

\subsection{Szolgáltatások:}\label{szolguxe1ltatuxe1sok}

A szolgáltatások meghatározásánál elsőnek azt vettem alapul, hogy milyen
feladatokat kell teljesítenie a rendszernek, majd az erőforrásokat
vettem alapul.

A könyvesbolthoz tartozóan a következő tevékenységeket határoztam meg: *
Bejelentkezés: Felhasználó felületen történő authentikálása * Böngészés:
Felhasználó láthatja mi van a raktáron * Vásárlás: Felhasználó valamit a
saját nevére ír Ezekből a feladatokból a következő szolgáltatásokat
lehet elkészíteni: * Felület kiszolgálása: Egy web kiszolgáló
alkalmazása, amin keresztül elvégezhetők a különböző műveletek, mint a
bejeletkezés, vagy vásárlás. Ez a felület magába foglalja a böngészést
lehetővé tevő szolgáltatást is. * Authentikációs szolgáltatás: A
bejelentkezni szándékozó felhasználó adatait ellenőrzi, és hibás
bejelentkezés esetén hibát dob. * Vásárlási szolgáltatás: A böngészés
közben kiválasztott könyveket lefoglalja a raktári készletből. *
Adatbázis szolgáltatás: Ez a szolgáltatás tartalmazza a raktár
tartalmát, a vásárlási naplót, és a bejelentkezési adatokat. * Terhelés
elosztó szolgáltatás: Ez a szolgálatatás a skálázhatóságot segíti, és
egy egységes interfész kialakításában segít.

\subsection{Megvalósítás:}\label{megvaluxf3suxedtuxe1s}

A megvalósításhoz felhasznált technológiák a szolgálatatások
felismerésében különböztek. Kipróbáltam a korábbi félévek során használt
Consul-t, amivel dinamikusan esemény vezérelten képesek kommunikálni a
szolgálatatások. Másodszorra a Docker konténerekbe beépített módzsert
használtam fel, amivel könnyen, már indítás közben felismerik egymást a
szolgáltatások. Harmadszorra pedog egy gyakran használt service
registry-t használtam, az Apache Zookeepert.

\subsection{Megvalósítás Docker
konténerekkel:}\label{megvaluxf3suxedtuxe1s-docker-kontuxe9nerekkel}

Az egyszerűség kedvéért, és a koncepció kipróbálásához Docker
konténereket használtam, mivel ezek könnyedén elindíthatók,
kkonfigurálhatók, és helyi gépen is lehetővé teszik egy komplex
architektúra kipróbálását.

A mikro szolgáltatások egyik legnagyobb előnye, hogy különböző
platformokat és programozási nyelveket használhatunk az architektúrában
különösebb probléma nélkül. Ezt a Docker-el úgy oldottam meg, hogy
Centos és Ubuntu disztribúciójú környezeteket, és PHP, Python, Java,
illetve Bash szkripteket használtam.

A szolgáltatásokhoz tartozó Docker konténerek: * Adatbázis: Az alapja
egy `mysql' nevezetű konténer, ami tartalmaz egy lightweight Ubuntu-t és
benne telepítve egy mysql szervert. Ezt a konténert egy inicializáló
szkripttel egészítettem ki, ami elkészítette az alap adatbázist. *
Terhelés elosztó: A terhelés elosztást HAProxy-val oldottam meg, amit
egy Ubuntu konténerre alapoztam. Létezik egy olyan Docker konténer, ami
kifejezetten HAProxy mikro szolgáltatásnak van nevezve, azonban ez a
konténer nehezen használható, és a szolgáltatás újraindítása is el lett
rontva benne, így egyszerűbbnek láttam egy saját megvalósítást
használni. * Webkiszolgáló: A weboldal kiszolgálását egy `httpd' nevű
lightweight konténer szolgálja ki amiben egy apache webkiszlgáló van.
Ezt kiegészítettem PHP-val, és néhány szkripttel, ami kiszolgálja a
kéréseket. * Authentikáció: Egyszerű Ubuntu konténer, ami fel van
szerelve Python-nal, és a MySQLdb Python könyvtárral. Ezen felül
tartalmaz egy REST-es kiszolgálót, amin keresztül elérhető a
szolgáltatás. * Vásárlás: Centos konténer alapú környezet, amiben Java
lett telepítve, és egy webes REST API-n keresztül érhetjük el a
szolgáltatását.

\subsection{Kapcsolatok építése
Consul-al:}\label{kapcsolatok-uxe9puxedtuxe9se-consul-al}

A Consul alkalmazást korábbi félév folyamán használtam már, teljesítmény
mérések futtatására, így megpróbáltam átültetni a logikát a jelenlegi
mikro szolgáltatásokat biztosító architektúrába. A gondot az okozta,
hogy a Consul alkalmazásnak szükséges egy fix pont, és ehhez találnom
kellett egy olyan elemet, ami mindenképpen elsőnek indul el. Ez az elem
lett a proxy szerver, ami összefogja az elemeket. A korábbi félévben
használt kód megfelelő volt számomra, mivel nagyon hasonló minta
alkalmazást használtam a teljesítmény mérésekhez is.

Ez a megoldás nem elég elosztott a mikro szolgáltatások tekintetében,
azonban egy elág hatékony, és könnyen implementálható megoldás. A mikro
szolgáltatásokra épülő architektúréban jellemzően van egy Service
Registry elem, ami lehetővé teszi a szolgáltatások nyílvántartását, és
ez biztosíthatja a kapcsolatot is. A Consul ebben a kialakításban
pontosan így is működött, viszont található olyan eszköz amit
kifejezetten a szolgáltatásokhoz találtak ki. Ez lenne például az Apache
Zookeeper.

\subsection{Kapcsolatok építése
Docker-el:}\label{kapcsolatok-uxe9puxedtuxe9se-docker-el}

Ahogy korábban már említettem lehetőség van a Docker legújabb verzióiban
megadni ,hogy ez egyes konténerek milyen néven és milyen hálózaton
keresztül érhető el a többi konténer. A név beállításához a docker run
parancs --hostname paraméterét használhatjuk, míg a hálózat
definiálásához előbb létre kell hozni egy új Docker hálózatot docker
network create bookstore amire a konténerek tudnak csatlakozni a --net
kulcsszóval. Ennek segítségével elértem, hogy nagyon egyszerűen és egy
eszköz felhasználásával képesek legyenek látni egymást a szolgáltatások,
viszont egy nagy hátulütője van a megoldásnak, mégpedig az, hogy egy
gépen kell futnia az összes alkalmazásnak. Mivel ez egy mikro
szolgáltatásokra épülő architektúránál közel sem ideális, így ez csupán
fejlesztési, és reprezentatív jelleggel használható. (Mivel a labor
célja, hogy bemutassam az architektúra működését, ezért ez megfelel
számomra)

\subsection{Kapcsolatok építése
Zookeeper-el:}\label{kapcsolatok-uxe9puxedtuxe9se-zookeeper-el}

TODO: Kipróbálni a Zookepert

\subsection{Automatizálás:}\label{automatizuxe1luxe1s}

A mikro szolgáltatások architektúrájában a következő feladatokat lehet
automatizálni:

\begin{enumerate}
\def\labelenumi{\arabic{enumi}.}
\tightlist
\item
  Teszt alkalmazás build-elése: Gyakran van szükség a szolgáltatást
  futtató fájlok és egyéb tartalmak fordítására (C, Java, bináris kép
  fájlok frodítása) , és ezeket a forrásokat könnyedén elkészíthetjük
  automatizáltan is, mielőtt a környezetet összeépítenénk.
\item
  Teszt architektúra telepítése: Az egyes szolgáltatásokat egy felügyelt
  környezetbe helyezve valamilyen környezeti konfigurációval együtt
  telepíthetjük (esetünkben Docker konténerekbe csomagolhatjuk), és az
  így kialakuló architektúrát használhatjuk fel a céljaunkra.
  (Esetünkben kialakítunk egy könyvesboltot)
\item
  Teszt architektúra konfigurálása: Van, hogy telepítés után nem elég
  magára hagyni a rendszert, és használni a szolgáltatásokat, de
  szükséges különböző beállításokat végrehajtani, hogy a megfelelő módon
  működjön az alkalmazás. Ilyen feladat lehet a szolgáltatásokhoz
  tartozó registry frissítése, vagy a futtató gépeken a rendelkezésre
  állás javítása, és egyéb biztonsági mechanizmusok alkalmazása.
  (Esetemben a Zookeeper felkonfigurálása lesz a feladat.)
\item
  Teszt architektúra tesztelése: Az éles futó architektúrán futtathatunk
  teszteket, amikkel megbizonyosodhatunk, hogy a rendszer megfelelően
  működik, és minden rendben van, átadható a megrendelőnek, vagy
  átengedhető a felhasználóknak. Ilyen teszt lehet az alkalmazás
  elemeinek a unit tesztelése, szolgáltatásonként funkció tesztek
  futtatása, a szolgáltatások kapcsolaihoz integrációs és rendszer
  tesztek futtatása, illetve a skálázás és egyéb teljesítményt
  befojásoló tényezőkhöz teljesítmény tesztek futtatása. (Esetemben unit
  teszteket fogok futtatni)
\end{enumerate}

\subsection{Jenkins Job-ok
fejelesztése:}\label{jenkins-job-ok-fejelesztuxe9se}

\subsection{Egyéb minta
alkalmazások:}\label{egyuxe9b-minta-alkalmazuxe1sok}

KanBan board minta:

https://github.com/eventuate-examples/es-kanban-board

Archivematica minta:

https://www.archivematica.org/en/

\end{document}
