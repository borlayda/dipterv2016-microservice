\appendix

\chapter{Függelék}\label{fuxfcggeluxe9k}

\section{Dockerfile-ok}\label{dockerfile-ok}

\subsection{Authentikáció}\label{authentikuxe1ciuxf3}

Dockerfile.auth.service

\begin{verbatim}
FROM ubuntu
MAINTAINER Borlay Dániel <borlay.daniel@gmail.com>

COPY auth.sh /usr/sbin/auth.sh
COPY auth-service.py /usr/sbin/auth-service.py

RUN apt-get -y update
RUN apt-get -y install vim bash python-oauth python-mysqldb python \
    python-flask
RUN chmod +x /usr/sbin/auth.sh
RUN chmod +x /usr/sbin/auth-service.py

EXPOSE 8081
\end{verbatim}

\subsection{Proxy}\label{proxy}

Dockerfile.proxy.service

\begin{verbatim}
FROM haproxy
MAINTAINER Borlay Dániel <borlay.daniel@gmail.com>

COPY proxy.sh /usr/sbin/proxy.sh

RUN apt-get -y update
RUN apt-get -y install vim haproxy
RUN chmod +x /usr/sbin/proxy.sh
COPY haproxy.cfg /etc/haproxy/haproxy.cfg

ENTRYPOINT proxy.sh

EXPOSE 8080
\end{verbatim}

\subsection{Adatbázis}\label{adatbuxe1zis}

Dockerfile.database.service

\begin{verbatim}
FROM ubuntu
MAINTAINER Borlay Dániel <borlay.daniel@gmail.com>

COPY database.sh /usr/sbin/database.sh
COPY auth_init.sql /tmp/auth_init.sql
COPY bookstore_init.sql /tmp/bookstore_init.sql

RUN apt-get -y update
RUN apt-get -y install mysql-server mysql-client vim
RUN chmod +x /usr/sbin/database.sh
RUN sed -i 's/bind-address.*=.*/bind-address = 0.0.0.0/g' /etc/mysql/my.cnf

EXPOSE 3306
\end{verbatim}

\subsection{Vásárlás}\label{vuxe1suxe1rluxe1s}

Dockerfile.reserve.service

\begin{verbatim}
FROM centos
MAINTAINER Borlay Dániel <borlay.daniel@gmail.com>

COPY reserve.sh /usr/sbin/reserve.sh

RUN yum -y update
RUN yum -y install vim java-1.8.0-openjdk-devel tomcat7
RUN chmod +x /usr/sbin/reserve.sh

EXPOSE 8888
\end{verbatim}

\subsection{Webkiszolgáló
(böngészés)}\label{webkiszolguxe1luxf3-buxf6nguxe9szuxe9s}

Dockerfile.webserver.service

\begin{verbatim}
FROM httpd
MAINTAINER Borlay Dániel <borlay.daniel@gmail.com>

COPY webserver.sh /usr/sbin/webserver.sh
COPY index.html /var/www/html/index.html
COPY login.php /var/www/html/login.php
COPY store.php /var/www/html/store.php

RUN apt-get -y update
RUN apt-get -y install vim php5 php5-mysql curl php5-curl
RUN chmod +x /usr/sbin/webserver.sh

EXPOSE 80 443
\end{verbatim}

\section{Szkriptek}\label{szkriptek}

\subsection{Futtatáshoz}\label{futtatuxe1shoz}

\subsubsection{Build}\label{build}

build\_docker.sh

\begin{verbatim}
#!/bin/bash

services="database webserver proxy reserve auth"

for service in ${services}
do
    echo "Create ${service} for bookstore ..."
    mkdir -p services/${service}
    cp Dockerfiles/Dockerfile.${service}.service services/${service}/Dockerfile
    cp -R scripts/${service}/* services/${service}/
    docker build -t bookstore_${service} services/${service} \
      &> services/${service}/build.log
done

echo "Microservices has been created!"
\end{verbatim}

\subsubsection{Futtatás}\label{futtatuxe1s}

run\_containers.sh

\begin{verbatim}
#!/bin/bash

services="database webserver reserve auth proxy"

docker network create bookstore

for service in ${services}
do
    echo "Start ${service} service ..."
    docker run -d --name "${service}" -h "${service}" \
      --net=bookstore bookstore_${service} ${service}.sh "${DOCKER_IP_HAPROXY}"
done
\end{verbatim}

\subsubsection{Tisztogatás}\label{tisztogatuxe1s}

clean\_docker.sh

\begin{verbatim}
#!/bin/bash

services="database webserver proxy reserve auth"

docker stop $(docker ps -a | awk '/bookstore/ {print $1}')
docker rm $(docker ps -a | awk '/bookstore/ {print $1}')

for service in ${services}
do
    echo "Delete ${service} image"
    docker rmi bookstore_${service}
done

rm -rf services
docker network rm bookstore
\end{verbatim}

\subsection{Szolgáltatásokhoz}\label{szolguxe1ltatuxe1sokhoz}

\subsubsection{Adatbázis
inicializálás}\label{adatbuxe1zis-inicializuxe1luxe1s}

Authentikáció:

\begin{verbatim}
# Add permission to databases
GRANT ALL PRIVILEGES ON authenticate.* TO 'root'@'%';
GRANT ALL PRIVILEGES ON authenticate.* TO 'root'@'localhost';
# Create Tables
CREATE TABLE user_auth
(
    user_id int NOT NULL AUTO_INCREMENT,
    username varchar(255) NOT NULL,
    password varchar(255) NOT NULL,
    credential varchar(255),
    PRIMARY KEY (user_id)
);
# Fill Tables
INSERT INTO user_auth (username, password) VALUES ("test", "testpassword");
\end{verbatim}

Bookstore raktár:

\begin{verbatim}
# Add permission to databases
GRANT ALL PRIVILEGES ON bookstore.* TO 'root'@'%';
GRANT ALL PRIVILEGES ON bookstore.* TO 'root'@'localhost';
# Create Tables
CREATE TABLE store
(
    store_id int NOT NULL AUTO_INCREMENT,
    book_name varchar(255) NOT NULL,
    count int NOT NULL,
    PRIMARY KEY (store_id)
);
CREATE TABLE reservation
(
    reservation_id int NOT NULL AUTO_INCREMENT,
    username varchar(255) NOT NULL,
    book_name varchar(255) NOT NULL,
    count int NOT NULL,
    res_date varchar(255),
    PRIMARY KEY (reservation_id)
);
# Fill Tables
INSERT INTO store (book_name, count)
VALUES ("Harry Potter and the Goblet of fire", 10);
INSERT INTO store (book_name, count)
VALUES ("Harry Potter and the Philosopher's Stone", 10);
INSERT INTO store (book_name, count)
VALUES ("Harry Potter and the Chamber of Secret", 10);
INSERT INTO store (book_name, count)
VALUES ("Lord of the Rings: Fellowship of the ring", 3);
INSERT INTO store (book_name, count)
VALUES ("Lord of the Rings: The Two Towers", 3);
INSERT INTO store (book_name, count)
VALUES ("Lord of the Rings: The Return of the King", 0);
\end{verbatim}

\subsubsection{Bejelentkezéshez}\label{bejelentkezuxe9shez}

login.php:

\begin{verbatim}
<?php

if(!isset( $_POST['username'], $_POST['password']))
{
    echo 'Please enter a valid username and password';
}
else
{
    $username = filter_var($_POST['username'], FILTER_SANITIZE_STRING);
    $password = filter_var($_POST['password'], FILTER_SANITIZE_STRING);

    $ch = curl_init();
    curl_setopt($ch, CURLOPT_RETURNTRANSFER, true);
    curl_setopt($ch, CURLOPT_URL,
        "http://auth:8081/auth/{$username}/{$password}"
    );
    $content = curl_exec($ch);
    echo $content;
}
?>
\end{verbatim}

auth-service.py:

\begin{verbatim}
#!/usr/bin/env python
from flask import Flask, abort
import MySQLdb as mdb
app = Flask(__name__)

@app.route("/auth/<username>/<password>")
def hello(username, password):
    try:
        con = mdb.connect('database', 'root', '', 'authenticate');
        cur = con.cursor()
        cur.execute("SELECT user_id FROM user_auth \
                     WHERE username='%s' AND password='%s'" %
                     (username, password))
        user_id = cur.fetchone()
        print user_id
        if not user_id:
            abort(401)
    except mdb.Error, e:
        print "Error %d: %s" % (e.args[0],e.args[1])
        abort(401)
    finally:
        if con:
            con.close()
    return "Successfully authenticated!"

if __name__ == "__main__":
    app.run(host="0.0.0.0", port=8081)
\end{verbatim}

\subsubsection{Böngészés}\label{buxf6nguxe9szuxe9s}

index.html:

\begin{verbatim}
<html>
<head>
<title>Bookstore Microservice</title>
</head>
<body>

<h2>Login:</h2>
<form action="login.php" method="post">
  <fieldset>
    <p>
      <label for="username">Username</label>
      <input type="text" id="username" name="username" value=""/>
    </p>
    <p>
      <label for="password">Password</label>
      <input type="text" id="password" name="password" value=""/>
    </p>
    <p>
      <input type="submit" value="Login" />
    </p>
  </fieldset>
</form>

</body>
</html>
\end{verbatim}

store.php:

\begin{verbatim}
<html>
<head>
<title>Bookstore Microservice</title>
</head>
<body>

<h2>Books:</h2>

<table>
  <tbody>
    <tr><th>Name</th><th>Quantity</th></tr>

    <?php
      $servername = "database";
      $username = "root";
      $password = "";
      $dbname = "bookstore";

      // Create connection
      $conn = new mysqli($servername, $username, $password, $dbname);
      // Check connection
      if ($conn->connect_error) {
          die("Connection failed: " . $conn->connect_error);
      }

      $sql = "SELECT * FROM store";
      $result = $conn->query($sql);

      if ($result->num_rows > 0) {
          // output data of each row
          while($row = $result->fetch_assoc()) {
              echo "<tr><td>" . $row["book_name"]. \
              "</td><td> " . $row["count"]. "</td></tr>";
          }
      } else {
          echo "0 results";
      }
      $conn->close();
    ?>

  </tbody>
</table>

</body>
</html>
\end{verbatim}

Proxy config:

\begin{verbatim}
...
frontend web
    bind *:80
    mode http
    default_backend nodes

backend nodes
    mode http
    balance roundrobin
    server webserver webserver:80 cookie check
\end{verbatim}
