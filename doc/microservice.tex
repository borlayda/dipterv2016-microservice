\documentclass[11pt,magyar,a4paper,oneside,]{report}
\usepackage[T1]{fontenc}
\usepackage{ae}
\usepackage{lmodern}
\usepackage{amssymb}
\usepackage{amsmath}
\usepackage{ifxetex,ifluatex}
\usepackage{fixltx2e} % provides \textsubscript
\usepackage{adjustbox}
\usepackage[thmmarks]{ntheorem}
\usepackage{listings}
\usepackage{color}
\usepackage{lastpage}
\usepackage{anysize}
\usepackage{longtable}
\usepackage{sectsty}
\usepackage{setspace}
\usepackage[hang]{caption}
\usepackage{tabularx}
\usepackage{hyphenat}
\usepackage{enumitem}
\usepackage{subcaption}
\usepackage{todonotes}
\usepackage{adjustbox}
\usepackage{minibox}
\usepackage{pdfpages}
\usepackage{tikz}
% fix for pandoc 1.14
\providecommand{\tightlist}{%
  \setlength{\itemsep}{0pt}\setlength{\parskip}{0pt}}
% use microtype if available
\IfFileExists{microtype.sty}{\usepackage{microtype}}{}
% use upquote if available, for straight quotes in verbatim environments
\IfFileExists{upquote.sty}{\usepackage{upquote}}{}
\ifnum 0\ifxetex 1\fi\ifluatex 1\fi=0 % if pdftex
  \usepackage[utf8]{inputenc}
\else % if luatex or xelatex
  \usepackage{fontspec}
  \ifxetex
    \usepackage{xltxtra,xunicode}
  \fi
  \defaultfontfeatures{Mapping=tex-text,Scale=MatchLowercase}
  \newcommand{\euro}{€}
\fi
\usepackage{natbib}
\bibliographystyle{plain}
\usepackage{color}
\usepackage{fancyvrb}
\newcommand{\VerbBar}{|}
\newcommand{\VERB}{\Verb[commandchars=\\\{\}]}
\DefineVerbatimEnvironment{Highlighting}{Verbatim}{commandchars=\\\{\}}
% Add ',fontsize=\small' for more characters per line
\newenvironment{Shaded}{}{}
\newcommand{\KeywordTok}[1]{\textcolor[rgb]{0.00,0.44,0.13}{\textbf{{#1}}}}
\newcommand{\DataTypeTok}[1]{\textcolor[rgb]{0.56,0.13,0.00}{{#1}}}
\newcommand{\DecValTok}[1]{\textcolor[rgb]{0.25,0.63,0.44}{{#1}}}
\newcommand{\BaseNTok}[1]{\textcolor[rgb]{0.25,0.63,0.44}{{#1}}}
\newcommand{\FloatTok}[1]{\textcolor[rgb]{0.25,0.63,0.44}{{#1}}}
\newcommand{\CharTok}[1]{\textcolor[rgb]{0.25,0.44,0.63}{{#1}}}
\newcommand{\StringTok}[1]{\textcolor[rgb]{0.25,0.44,0.63}{{#1}}}
\newcommand{\CommentTok}[1]{\textcolor[rgb]{0.38,0.63,0.69}{\textit{{#1}}}}
\newcommand{\OtherTok}[1]{\textcolor[rgb]{0.00,0.44,0.13}{{#1}}}
\newcommand{\AlertTok}[1]{\textcolor[rgb]{1.00,0.00,0.00}{\textbf{{#1}}}}
\newcommand{\FunctionTok}[1]{\textcolor[rgb]{0.02,0.16,0.49}{{#1}}}
\newcommand{\RegionMarkerTok}[1]{{#1}}
\newcommand{\ErrorTok}[1]{\textcolor[rgb]{1.00,0.00,0.00}{\textbf{{#1}}}}
\newcommand{\NormalTok}[1]{{#1}}
\usepackage{longtable,booktabs}
\usepackage{graphicx}
% We will generate all images so they have a width \maxwidth. This means
% that they will get their normal width if they fit onto the page, but
% are scaled down if they would overflow the margins.
\makeatletter
\def\maxwidth{\ifdim\Gin@nat@width>\linewidth\linewidth
\else\Gin@nat@width\fi}
\makeatother
\let\Oldincludegraphics\includegraphics
%\renewcommand{\includegraphics}[1]{\Oldincludegraphics[scale=1.0]{#1}}
\renewcommand{\includegraphics}[1]{
\begin{adjustbox}{max size={\textwidth}{\textheight}}
    \Oldincludegraphics[scale=0.6]{#1}%
\end{adjustbox}
}
\ifxetex
  \usepackage[setpagesize=false, % page size defined by xetex
              unicode=false, % unicode breaks when used with xetex
              xetex]{hyperref}
\else
  \usepackage[unicode=true]{hyperref}
\fi
\definecolor{darkgreen}{rgb}{0,0.5,0}
\hypersetup{breaklinks=true,
            bookmarks=true,
            pdfauthor={},
            pdftitle={},
            colorlinks=true,
            urlcolor=blue,
            linkcolor=magenta,
            citecolor=darkgreen,
            pdfborder={0 0 0}}
\urlstyle{same}  % don't use monospace font for urls
\setlength{\parindent}{0pt}
\setlength{\parskip}{6pt plus 2pt minus 1pt}
\setlength{\emergencystretch}{3em}  % prevent overfull lines
\ifxetex
  \usepackage{polyglossia}
  \setmainlanguage{magyar}
\else
  \usepackage[magyar]{babel}
\fi

\title{Mikroszolgáltatásokra épülő architektúra fejlesztésének és tesztelésének támogatása}
\author{Borlay Dániel}

\renewcommand*{\hyperref}[2][\ar]{%
  \def\ar{#2}
  #2 (\refstruc{#1})}

\renewcommand*{\figureautorefname}{ábra}

\sloppy

% for English documents
%\setlength{\parindent}{0pt} % áttekinthetőbb, angol nyelvű dokumentumokban jellemző
%\setlength{\parskip}{8pt plus 3pt minus 3pt} % áttekinthetőbb, angol nyelvű dokumentumokban jellemző

% for Hungarian documents
\setlength{\parindent}{12pt}
\setlength{\parskip}{0pt}

\marginsize{35mm}{25mm}{15mm}{15mm} % anysize package
\setcounter{secnumdepth}{0}
\sectionfont{\large\upshape\bfseries}
\setcounter{secnumdepth}{3}
\singlespacing
\frenchspacing


\begin{document}

\footnotesize

\textbf{Általános információk, a diplomaterv szerkezete}

A diplomaterv szerkezete a BME Villamosmérnöki és Informatikai Karán:

\begin{enumerate}
\def\labelenumi{\arabic{enumi}.}
\itemsep1pt\parskip0pt\parsep0pt
\item
  Diplomaterv feladatkiírás
\item
  Címoldal
\item
  Tartalomjegyzék
\item
  A diplomatervező nyilatkozata az önálló munkáról és az elektronikus
  adatok kezeléséről
\item
  Tartalmi összefoglaló magyarul és angolul
\item
  Bevezetés: a feladat értelmezése, a tervezés célja, a feladat
  indokoltsága, a diplomaterv felépítésének rövid összefoglalása
\item
  A feladatkiírás pontosítása és részletes elemzése
\item
  Előzmények (irodalomkutatás, hasonló alkotások), az ezekből levonható
  következtetések
\item
  A tervezés részletes leírása, a döntési lehetőségek értékelése és a
  választott megoldások indoklása
\item
  A megtervezett műszaki alkotás értékelése, kritikai elemzése,
  továbbfejlesztési lehetőségek
\item
  Esetleges köszönetnyilvánítások
\item
  Részletes és pontos irodalomjegyzék
\item
  Függelék(ek)
\end{enumerate}

Felhasználható a következő oldaltól kezdődő \LaTeX-diplomaterv sablon
dokumentum tartalma.

A diplomaterv szabványos méretű A4-es lapokra kerüljön. Az oldalak
tükörmargóval készüljenek (mindenhol 2,5 cm, baloldalon 1 cm-es
kötéssel). Az alapértelmezett betűkészlet a 12 pontos Times New Roman,
másfeles sorközzel.

Minden oldalon -- az első négy szerkezeti elem kivételével --
szerepelnie kell az oldalszámnak.

A fejezeteket decimális beosztással kell ellátni. Az ábrákat a megfelelő
helyre be kell illeszteni, fejezetenként decimális számmal és kifejező
címmel kell ellátni. A fejezeteket decimális aláosztással számozzuk,
maximálisan 3 aláosztás mélységben (pl. 2.3.4.1.). Az ábrákat,
táblázatokat és képleteket célszerű fejezetenként külön számozni (pl.
2.4. ábra, 4.2. táblázat. vagy képletnél (3.2)). A fejezetcímeket
igazítsuk balra, a normál szövegnél viszont használjunk
sorkiegyenlítést. Az ábrákat, táblázatokat és a hozzájuk tartozó címet
igazítsuk középre. A cím a jelölt rész alatt helyezkedjen el.

A képeket lehetőleg rajzoló programmal készítsék el, az egyenleteket
egyenlet-szerkesztő segítségével írják le (A \LaTeX~ehhez kézenfekvő
megoldásokat nyújt).

Az irodalomjegyzék szövegközi hivatkozása történhet a
Harvard-rendszerben (a szerző és az évszám megadásával) vagy
sorszámozva. A teljes lista névsor szerinti sorrendben a szöveg végén
szerepeljen (sorszámozott irodalmi hivatkozások esetén hivatkozási
sorrendben). A szakirodalmi források címeit azonban mindig az eredeti
nyelven kell megadni, esetleg zárójelben a fordítással. A listában
szereplő valamennyi publikációra hivatkozni kell a szövegben (a
\LaTeX-sablon a Bib\TeX~segítségével mindezt automatikusan kezeli).
Minden publikáció a szerzők után a következő adatok szerepelnek:
folyóirat cikkeknél a pontos cím, a folyóirat címe, évfolyam, szám,
oldalszám tól-ig. A folyóirat címeket csak akkor rövidítsük, ha azok
nagyon közismertek vagy nagyon hosszúak. Internet hivatkozások
megadásakor fontos, hogy az elérési út előtt megadjuk az oldal
tulajdonosát és tartalmát (mivel a link egy idő után akár elérhetetlenné
is válhat), valamint az elérés időpontját.

Fontos:

\begin{itemize}
\itemsep1pt\parskip0pt\parsep0pt
\item
  A szakdolgozat készítő/diplomatervező nyilatkozata (a jelen sablonban
  szereplő szövegtartalommal) kötelező előírás Karunkon ennek hiányában
  a szakdolgozat/diplomaterv nem bírálható és nem védhető!
\item
  Mind a dolgozat, mind a melléklet maximálisan 15 MB méretű lehet!
\end{itemize}

Jó munkát, sikeres szakdolgozat készítést, ill. diplomatervezést
kívánunk!

\clearpage

\textbf{Feladatkiírás}

A feladatkiírást a tanszéki adminisztrációban lehet átvenni, és a
leadott munkába eredeti, tanszéki pecséttel ellátott és a tanszékvezető
által aláírt lapot kell belefűzni (ezen oldal \emph{helyett}, ez az
oldal csak útmutatás). Az elektronikusan feltöltött dolgozatba már nem
kell beleszerkeszteni ezt a feladatkiírást.


\normalsize

%--------------------------------------------------------------------------------------
% The title page
%--------------------------------------------------------------------------------------
\begin{titlepage}
\begin{center}
\Oldincludegraphics[width=60mm,keepaspectratio]{img/BME1782logo.pdf}\\

\vspace{0.3cm}
\textbf{Budapesti Műszaki és Gazdaságtudományi Egyetem}\\
\textmd{Méréstechnika és Információs Rendszerek Tanszék}\\
\textmd{}\\[5cm]

\vspace{0.4cm}
{\huge \bfseries Mikroszolgáltatásokra épülő architektúra fejlesztésének és tesztelésének támogatása}\\[0.8cm]
\vspace{0.5cm}
\textsc{\Large Diplomaterv}\\[4cm]

\begin{tabular}{cc}
 \makebox[7cm]{\emph{Készítette}} & \makebox[7cm]{\emph{Konzulens}} \\
 \makebox[7cm]{Borlay Dániel} & \makebox[7cm]{Szatmári Zoltán}
\end{tabular}

\vfill
{\large \today}
\end{center}
\end{titlepage}

\onehalfspacing

\hypersetup{linkcolor=black}
\setcounter{tocdepth}{3}
\tableofcontents

\vfill
\clearpage

\begin{center}
\large
\textbf{HALLGATÓI NYILATKOZAT}\\
\end{center}

Alulírott \emph{Borlay Dániel}, szigorló hallgató kijelentem, hogy ezt a diplomatervet meg nem engedett segítség nélkül, saját magam készítettem, csak a megadott forrásokat (szakirodalom, eszközök stb.) használtam fel. Minden olyan részt, melyet szó szerint, vagy azonos értelemben, de átfogalmazva más forrásból átvettem, egyértelműen, a forrás megadásával megjelöltem.

Hozzájárulok, hogy a jelen munkám alapadatait (szerző(k), cím, angol és magyar nyelvű tartalmi kivonat, készítés éve, konzulens(ek) neve) a BME VIK nyilvánosan hozzáférhető elektronikus formában, a munka teljes szövegét pedig az egyetem belső hálózatán keresztül (vagy autentikált felhasználók számára) közzétegye. Kijelentem, hogy a benyújtott munka és annak elektronikus verziója megegyezik. Dékáni engedéllyel titkosított diplomatervek esetén a dolgozat szövege csak 3 év eltelte után válik hozzáférhetővé.

\begin{flushleft}
\vspace*{1cm}
Budapest, \today
\end{flushleft}

\begin{flushright}
 \vspace*{1cm}
 \makebox[7cm]{\rule{6cm}{.4pt}}\\
 \makebox[7cm]{\emph{Borlay Dániel}}\\
 \makebox[7cm]{hallgató}
\end{flushright}
\thispagestyle{empty}

\vfill
\clearpage
\thispagestyle{empty} % an empty page

\chapter*{Kivonat}\label{kivonat}
\addcontentsline{toc}{chapter}{Kivonat}

Jelen dokumentum egy diplomaterv sablon, amely formai keretet ad a BME
Villamosmérnöki és Informatikai Karán végző hallgatók által elkészítendő
szakdolgozatnak és diplomatervnek. A sablon használata opcionális. Ez a
sablon Markdown leírónyelven készült, Pandoc rendszerrel fordítható le
\TeX~Live vagy MiK\TeX~\LaTeX~disztribúciókkal.

\chapter*{Abstract}\label{abstract}
\addcontentsline{toc}{chapter}{Abstract}

This document is a skeleton for BSc/MSc~theses of students at the
Electrical Engineering and Informatics Faculty, Budapest University of
Technology and Economics. The usage of this skeleton is optional. The
skeleton was implemented in Markdown and can be compiled with Pandoc,
using the \TeX~Live and or the MiK\TeX~\LaTeX~compiler.

\chapter{Bevezetés}\label{bevezetuxe9s}

A bevezető tartalmazza a diplomaterv-kiírás elemzését, történelmi
előzményeit, a feladat indokoltságát (a motiváció leírását), az eddigi
megoldásokat, és ennek tükrében a hallgató megoldásának összefoglalását.

A bevezető szokás szerint a diplomaterv felépítésével záródik, azaz
annak rövid leírásával, hogy melyik fejezet mivel foglalkozik.

\chapter{Mikro szolgáltatások}\label{mikro-szolguxe1ltatuxe1sok}

\subsection{Definíció:}\label{definuxedciuxf3}

Nem találtam konkrét definicót, de a mikro szolgáltatás egy olyan
architektúrális modellezési mód, amikor a tervezett
rendszert/alkalmazást kisebb funkciókra bontjuk, és önálló
szolgáltatásokként, önálló erőforrásokkal, valamilyen jól definiált
interfészen keresztül tesszük elérhetővé.

\subsection{Technológiáról:}\label{technoluxf3giuxe1ruxf3l}

A mikro szolgáltatás architektúra kiépítéséhez sokféle szétválasztási
módot használnak, amik közül van olyan amit a tervezési folyamat közben
felmerülő főneveket, vagy igéket használják fel, de abban megegyeznek,
hogy a funkcionlaitást bontják fel. Ezzekkel az
\href{Integrációs-minták}{integrációval foglalkozó részben} olvashatunk
bővebben.

A mikro szolgáltatások tervezése során a következő szempontok szerint
szokták megtervezni a rendszert:

\begin{itemize}
\itemsep1pt\parskip0pt\parsep0pt
\item
  Milyen szolgáltatásokat tud nyújtani a rendszer
\item
  Lehetséges műveletek felsorolása (igék amik a rendszerrel
  kapcsolatosak)
\item
  Lehetséges erőforrások vagy entitások felsorolása (főnevek alapján
  szétválasztás)
\item
  Lehetséges use-case-ek szétválasztása (felhasználási módszerek
  elválasztása)
\item
  A felbontott rendszert hogyan kapcsoljuk össze
\item
  Pipeline-ként egy hosszú folyamatot összeépítve és az információt
  áramoltatva
\item
  Elosztottan, igény szerint meghívva az egyes szolgáltatásokat
\item
  Egyes funkciókat összekapcsolva nagyobb szolgáltatások kialakítása
  (kötegelés)
\item
  A kommunikáció a felhasználóval
\item
  Egy központi szolgáltatáson keresztül, ami a többivel kommunikál
\item
  Add-hoc minden szolgáltatás külön hívható
\end{itemize}

Ezekkel a lépéssekkel meg lehet alapozni, hogy az álltalunk készítendő
rendszer hogyan is lesz kialakítva, és milyen paraméterek mentén lesz
felvágva. A választást segíti a témában elterjedt fogalom, a scaling
cude\href{http://microservices.io/articles/scalecube.html}{{[}1{]}}, ami
azt mutatja, hogy az architektúrális terveket milyen szempontok mentén
lehet felosztani.

\begin{figure}[htbp]
\centering
\includegraphics{http://microservices.io/i/DecomposingApplications.021.jpg}
\caption{Scaling Cube}
\end{figure}

Ahogy a képen is látható a meghatározó felbontási fogalmak, az adat
menti felbontás, a tetszőleges fogalom menti felbontás, illetve a
klónozás.

Az adat menti felbontás annyit tesz, hogy a szolgáltatásokat annak
megfelelően bontjuk fel, hogy az egyes szolgáltatások csak adatbázissal,
vagy csak web kiszolgálással foglalkozzanak, vagy csak a felhasználói
adatok esetleg a tanulók jegyeit felügyelik. Ez a mérce a mikro
szolgáltatás architektúrák esetén nem annyira fontos, mivel a
szolgáltatásoknak erőforrásaikat tekintve is el kell különülniük, így
nem éri meg erőforrások vagy adat mentén vágni.

A tetszőleges fogalom menti felbontás annyit tesz hogy elosztott
rendszert hozunk létre tetszőleges funkcionalitás szerint. Erre épít a
mikro szolgáltatás architektúra is, mivel a lényege pont az egyes
funkciók atomi felbontása.

A harmadik módszer arra tér ki, hogy hogyan lehet egy architektúrát
felosztani, hogy skálázható legyen. Itt a klónozhatóság, avagy az egymás
melleti kiszálgálás motivál. Ez a mircro-service-eknél kell, hogy
teljesüljön, mivel adott esetben a load balancer alatt tudnunk kell
definiálni több példányt is egy szolgáltatásból.

\section{Architektúrális minták}\label{architektuxfaruxe1lis-mintuxe1k}

Mint korábban láthattuk vannak bizonyos telepítési módszerek, amik
mentén szokás a mikro szolgáltatásokat felépíteni. Van aki az
architektúrális tervezési minták közé sorolja a mikro szolgáltatás
architektúrát, azonban nem lehet élesen elkülöníteni, mivel valamilyen
csatolási módszerre szükség van, ami nem specifikus a mikro
szolgáltatás-ek esetén, viszont más architektúrális mintákra jellemző.

Ilyen a Pipes and fileter architektúrális minta
\href{https://msdn.microsoft.com/en-us/library/dn568100.aspx}{{[}2{]}},
aminek a lényege, hogy a funkciókra bontott architektúrát az elérni
kívánt végeredmény érdekében különböző módokon összekötünk. Ebben a
módszerben az adat folyamatosan áramlik az egyes alkotó elemek között,
és lépésről lépésre alakul ki a végeredmény. Elég olcsón kivitelezhető
architektúrális minta, mivel csupán sorba kell kötni hozzá az egyes
szolgáltatásokat, azonban nehezen lehet optimalizálni, és könnyen lehet,
hogy olyan részek lesznek a feldolgozás közben, amik hátráltatják a
teljes folyamatot.

Egy másik elosztott rendszerekhez kitallált minta a
subscriber/publisher\href{https://msdn.microsoft.com/en-us/library/ff649664.aspx}{{[}3{]}},
amely arra alapszik, hogy egy szolgáltatásnak szüksége van valamilyan
adatra vagy funkcióra, és ezért feliratkozik egy másik szolgáltatásra.
Ennek az lesz az eredménye, hogy bizonyos szolgáltatások bizonyos más
szolgáltatásokhoz fognak kötődni, és annak megfelelően fognak egymással
kommunikálni, hogy milyen feladatot kell végrehajtaniuk.

\subsection{Hivatkozások:}\label{hivatkozuxe1sok}

{[}1{]} Chris Richardson:
\href{http://microservices.io/articles/scalecube.html}{The Scale Cude}

{[}2{]} Microsoft: {[}Pipes and Filters Pattern{]}
(https://msdn.microsoft.com/en-us/library/dn568100.aspx)

{[}3{]} Microsoft:
\href{https://msdn.microsoft.com/en-us/library/ff649664.aspx}{Publish/Subscribe}

\subsection{Előnyök-hátrányok:}\label{elux151nyuxf6k-huxe1truxe1nyok}

A mirco-service architektúrák a monolitikus architektúra ellentetjei,
melyben az erőforrások központilag vannak jezelve, és minden funkció egy
nagy interfészen keresztül érhető el. A monolitikus architektúra
egyszerűen kiépíthető, könnyű tervezni és fejleszteni, azonban nehezen
lehet kicserélni, nem elég robosztus, és nehezen skálázható, mivel az
erőforrásokat közösen kezelik a funkciók.

Ezzel ellenzétben a mikro szolgáltatás architektúrát ugyan nehezen lehet
megtervezni, hiszen egy elosztott rendszert kell megtervezni, ahol az
adatátviteltől kezdve az erőforrás megosztáson keresztül semmi sem
egyértelmű, viszont a későbbi tovább fejlesztés sokkal egyszerűbb, mivel
külön csapatokat lehet rendelni az egyes szolgáltatásokhoz, és könnyen
integrálhatók kicserélhetők az alkotó elemek. Mivel sok kis egységből
áll, könnyebben lehet úgy skálázni a rendszert, hogy ne pazaroljuk el az
erőforrásainkat, és ugyanakkor a kis szolgáltatások erőforrásokban is el
vannak különítve, így nem okoz gondot, hogy fel vagy le skálázzunk egy
szolgáltatást. Ennek az a hátránya, hogy le kell kezelni a skálázáskor a
közös erőforrásokat.(Például ha veszünk egy autentikációs szolgáltatást,
akkor ha azt fel skálázzuk, meg kell tartanunk a felhasználók listáját,
így duplikálni kell az adatbázist, és fenntartani a konzisztenciát)
Ugyan csak előnye a mirco-service architektúrának, hogy különböző
technológiákat lehet keverni vele, mivel az egyes szolgáltatások
különböző technológiákkal különböző platformon is futhatnak.

\subsection{Kommunikáció:}\label{kommunikuxe1ciuxf3}

A szolgáltatások közötti kommunikáció nincs lekötve de jellemző a
REST-es API, vagy a webservice-re jellemző XML alapú kommunikáció.
Minden szolgáltatához tartozik egy önálló interfész, amin keresztül a
többi szolgáltatás kommunikálhat vele, és minden funkcióját el lehet
érni. Ennek az interfésznek olyannak kell lennie, hogy az implementáció
szabadon változtatható legyen, és ne kelljen más szolgáltatásokat
megváltoztatni, ha a saját szolgáltatásunkat változtatjuk. Ez segíti a
több csapattal való munkát, és lehetővé teszi hogy teljesen függetlenül
létezzenek a szolgáltatások.

\subsection{Példák:}\label{puxe9lduxe1k}

Amazon - minden Amazon-nal kommunikáló eszköz illetve az egyes funkciók
implementációja is szolgáltatásokra van szedve, és ezeket hívják az
egyes funkciók (vm indítás, törlés, mozgatás, stb.)

eBay - Különböző műveletek szerint van felbonva a a funkcionalitás, és
ennek megfelelően külön szolgáltatásként érhető el a fizetés,
megrendelés, szállítási információk, stb.

NetFlix - A nagy terhelést elkerülendő bizonyos streaming
szolgáltatásokat átlalakítottak, hogy a mikro szolgáltatás architektúra
szerint működjön.

Mintapéldák: http://eventuate.io/exampleapps.html

\subsection{Hivatkozások:}\label{hivatkozuxe1sok-1}

{[}1{]} Peter Van Garderen:
\href{http://citeseerx.ist.psu.edu/viewdoc/download?doi=10.1.1.384.7168\&rep=rep1\&type=pdf\#page=145}{\emph{Archivematica:
Using Micro-Services and open-source software to deliver a conprehensive
digital curation solution}}

{[}2{]} Manfred Sneps-Sneppe, Dmitry Namiot:
\href{http://injoit.ru/index.php/j1/article/view/161/119}{\emph{Micro-service
Architecture for Emerging Telecom Applications}}

{[}3{]} Chris Richardson (Kong):
\href{http://microservices.io/patterns/microservices.html}{\emph{Pattern:
Microservices Architecture}}

{[}4{]} Jose Ignacio Fernández-Villamor, Carlos Á. Iglesias, Mercedes
Garijo:
\href{http://oa.upm.es/8128/1/INVE_MEM_2010_81293.pdf}{\emph{Microservices:
Lightwieght services descriptors for REST architectural style}}

{[}5{]} Dimirty Namiot, Manfréd Sneps-Sneppe:
\href{http://cyberleninka.ru/article/n/on-micro-services-architecture}{\emph{On
Mirco-services Architecture}}

{[}6{]} Augusto Ciuffolettia:
\href{http://ac.els-cdn.com/S187705091503077X/1-s2.0-S187705091503077X-main.pdf?_tid=83f9a800-e3fa-11e5-9747-00000aacb35e\&acdnat=1457310260_fa7b3e651c221cf0307fbb2d6c7f59a6}{\emph{Automated
deployment of a microservice-based monitoring infrastructure}}

{[}7{]} Reagan Moore:
\href{http://ijdc.net/index.php/ijdc/article/view/63/42}{\emph{Towards a
Theory of Digital Preservation}}

{[}8{]} Sebastián Peyrott:
\href{https://auth0.com/blog/2015/09/04/an-introduction-to-microservices-part-1/}{An
Introduction to Microservices, Part 1}

{[}9{]} James Lewis, Martin Fowler:
\href{http://martinfowler.com/articles/microservices.html}{Microservices
a definition of this new architectural term}

\chapter{Technológiai
áttekintés}\label{technoluxf3giai-uxe1ttekintuxe9s}

Az integrációhoz olyan technológiákat lehet használni, melyek lehetővé
teszik az egyes szolgáltatások elkülönült működését.

A következő feladatokre kellenek technológiák: * Hogyan lehet
feltelepíteni egy önálló szolgáltatást? (telepítés) * Hogyan lehet
összekötni ezeket a szolgáltatásokat? (automatikus környezet felderítés)
* Hogyan lehet fenntartani, változtatni a szolgáltatások környezetét?
(integrációs keretrendszer) * Hogyan lehet skálázni a szolgáltatást?
(skálázás) * Hogyan lehet egységesen használni a skálázott
szolgáltatásokat? (load balance, konzisztencia fenntartás) * Hogyan
lehet virtualizáltan ezt kivitelezni? (virtualizálás) * A meglévő
szolgáltatásokat hogyan tartsuk nyilván? (service registy) * Hogyan
figyeljük meg a rendszert működés közben (monitorozás, loggolás)

\subsection{Telepítés:}\label{telepuxedtuxe9s}

A microservice-eket valamilyen módon létre kell hozni, egy hosthoz kell
rendelni, és az egyes elemeket össze kell kötni. A szolgáltatások
telepítéséhez olyan technológiára van szükség amivel könnyen elérhetünk
egy távoli gépet, és könnyen kezelhetsük az ottani erőforrásokat. Ehhez
a legkézenfekvőbb megoldás a Linux rendszerek esetén az SSH kapcsolaton
keresztül végrehajtott Bash parancs, de vannak eszközök, amikkel ezt
egyszerűbben és elosztottabban is megtehetjük.

\begin{itemize}
\item
  \textbf{Jenkins}: A Jenkins egy olyan folytonos integráláshoz
  kifejlesztett eszköz, mellyel képesek vagyunk különböző funkciókat
  automatizálni, vagy időzítetten futtani. A Jenkins egy Java alapú
  webes felülettel rendelkező alkalmazás, amely képes bash parancsokat
  futtatni, Docker konténereket kezelni, build-eket futtatni, illetve a
  hozzá fejlesztett plugin-eken keresztül, szinte bármire képes.
  Támogatja a fürtözést is, így képesek vagyunk Jenkins slave-eket
  létrehozni, amik a mester szerverrel kommunikálva végzik el a
  dolgukat. A microservice architektúrák esetén alkalmas a
  szolgáltatások telepítésére, és tesztelésére.
\item
  \textbf{ElasticBox}: Egy olyan alkalmazás, melyben nyilvántarthatjuk
  az alkalmazásainkat, és könnyen egyszerűen telepíthetjük őket.
  Támogatja a konfigurációk változását, illetve számos technológiát,
  amivel karban tarthatjuk a környezetünket. (Docker, Puppet, Ansible,
  Chef, stb) Együtt működik különböző cloud megoldásokkal, mint az AWS,
  vSphere, Azure, és más környezetek. Hasonlít a Jenkins-re, csupán ki
  van élezve a microservice architektúrák vezérlésére. (Illetve fizetős
  a Jenkins-el ellentétben) Mindent végre tud hajtani ami egy
  microservice alkalmazáshoz szükséges, teljes körű felügyeletet
  biztosít.
\end{itemize}

Egyéb lehetőség, hogy a fejlesztő készít magának egy olyan szkriptet,
ami elkészíti számára a micro-service architektúrát, és lehetővé teszi
az elemek dinamikus kicserélését. (ad-hoc megoldás)

\subsection{Környezet felderítés:}\label{kuxf6rnyezet-felderuxedtuxe9s}

Az egyes szolgáltatásoknak meg kell találniuk egymást, hogy megfelelően
működhessen a rendszer, azonban ez nem mindig triviális, így szükség van
egy olyan alkalmazásra, amivel felderíthetjük az aktív szolgáltatásokat.

\begin{itemize}
\itemsep1pt\parskip0pt\parsep0pt
\item
  \textbf{Consul}: A Hashicorp szolgáltatás felderítő alkalmazása, amely
  egy kliens-szerver architektúrának megfelelően megtalálja a
  környezetében lévő szolgáltatásokat, és figyeli az állapotukat (ha
  inaktívvá válik egy szolgáltatás a Consul észreveszi). Ez az
  alkalmazás egy folyamatosan választott mester node-ból és a többi
  slave node-ból áll. A mester figyeli az alárendelteket, és kezeli a
  kommunikációt. Egy új slave-et úgy tudunk felvenni, hogy a consul
  klienssel kapcsolódunk a mesterre. Ha automatizáltan tudjuk vezényelni
  a feliratkozást, egy nagyon erős eszköz kerül a kezünkbe, mivel
  eseményeket küldhetünk a szervereknek, és ezekre különböző feladatokat
  hajthatunk végre.
\end{itemize}

\subsection{Integrációs
keretrendszer:}\label{integruxe1ciuxf3s-keretrendszer}

A telepítéshez és a rendszer állapotának a fenntartásához egy olyan
eszköz kell, amivel gyorsan egyszerűen végrehajthatjuk a
változtatásainkat, és ha valamit változtatunk egy szolgáltatásban, akkor
az összes hozzá hasonló szolgáltatás értesüljön a változtatáról, vagy
hajtson végre ő maga is változtatást.

\begin{itemize}
\item
  \textbf{Puppet}: Olyan nyilt forrású megoldás, amellyel leírhatjuk
  objektum orientáltan, hogy milyen változtatásokat akarunk elérni, és a
  Puppet elvégzi a változtatásokat. Automatizálja a szolgáltatás
  változtatásának minden lépését, és egyszerű gyors megoldást
  szolgáltatat a komplex rendszerbe integráláshoz.
\item
  \textbf{Chef}: A Chef egy olyan konfiguráció menedzsment eszköz ami
  nagy mennyiségű szerver számítógépet képes kezelni, fürtözhető, és
  megfigyeli az alá szervezett szerverek állapotát. Tartja a kapcsolatot
  a gépekkel, és ha valamelyik konfiguráció nem felel meg a definiált
  repectkönynek (amiben definiálhatjuk az elvárt környezeti
  paramétereket) akkor változtatásokat indít be, és eléri, hogy a
  szerver a megfelelő konfigurációval rendelkezzen. Népszerű
  konfiguráció menedzsment eszköz, amiz könnyedén használhatunk
  integrációhoz, illetve a szolgáltatások cseréjéhez, és
  karbantartásához.
\item
  \textbf{Ansible}: A Chef-hez hasonlóan képes változtatásokat
  eszközölni a szerver gépeken egy SSH kapcsolaton keresztül, viszont a
  Chef-el ellentétben nem tartja a folyamatos kapcsolatot. Az Ansible
  egy tipikusan integrációs célokra kifejlesztett eszköz, amelyhez
  felvehetjük a gépeket, amiken valamilyen konfigurációs változtatást
  akarunk végezni, és egy ``playbook'' segítségével leírhatjuk milyen
  változásokat kell végrehajtani melyik szerverre. Könnyen irányíthatjuk
  vele a szolgáltatásokat, és definiálhatunk szolgáltatásonként egy
  playbook-ot ami mondjuk egy fürtnyi szolgáltatást vezérel. Ez az
  eszköz hasznos lehet, ha egy szolgáltatásnak elő akarjuk készíteni a
  környezetet.
\item
  \textbf{SaltStack}: A SaltStack nagyon hasonlít a Chef-re, mivel ez a
  termék is széleskörű felügyeletet, és konfiguráció menedzsment-et
  kínál számunkra, amit folyamatos kapcsolat fenntartással, és gyors
  kommunikációval ér el. Az Ansible-höz nagyon hasonlóan konfigurálható
  (nem lennék meglepve ha azt használná a háttérben), szintén ágens
  nélküli kapcsolatot tud létesíteni, és a Chef-hez hasonlóan több 10
  ezer gépet tud egyszerre karbantartani.
\end{itemize}

\subsection{Skálázás:}\label{skuxe1luxe1zuxe1s}

A microservice architektúrák egyik nagy előnye, hogy az egyes funkciókra
épülő szolgáltatásokat könnyedén lehet skálázni, mivel egy load
balancert használva csupán egy újabb gépet kell beszervezni, és máris
nagyobb terhelést is elbír a rendszer. Ahhoz hogy ezt kivitelezni
tudjuk, szükségünk van egy terhelés elosztóra, és egy olyan logikára,
ami képes megsokszorozni az erőforrásainkat. Cloud-os környezetben ez
könnyen kivitelezhető, egyébként hideg tartalékban tartott gépek
behozatalával elérhető. Sajnálatos módon általános célú skálázó eszköz
nincsen a piacon, viszont gyakran készítenek maguknak saját logikát a
nagyobb gyártók.

\begin{itemize}
\itemsep1pt\parskip0pt\parsep0pt
\item
  \textbf{Elastic Load Balancer}: Az Amazon AWS-ben az ELB avagy
  rugalmas terhelés elosztó az, ami ezt a célt szolgálja. Ennek a
  szolgáltatásnak az lenne a lényege, hogy segítse az Amazon Cloud-ban
  futó virtuális gépek hibatűrését, illtve egységbe szervezi a különböző
  elérhetőségi zónákban lévő gépeket, amivel gyorsabb elérést tudunk
  elérni. Mivel ez a szolgáltatás csupán az Amazon AWS-t felhasználva
  tud működni, nem megfelelő általános célra, azonban ha az Amazon
  Cloud-ban építjük fel a microservice architektúránkat, akkor erős
  eszköz lehet számunkra.
\end{itemize}

\subsection{Load balancing:}\label{load-balancing}

A microservice architektúrának egyik fontos eleme a terhelés elosztó,
vagy valamilyen fürtözést lehetővé tevő eszköz. Ez azért fontos, mert
egy egységes interfészt tudunk kialakítani a szolgáltatásaink elérésére,
és könnyíti a skálázódást a szolgáltatások mentén.

\begin{itemize}
\item
  \textbf{HAProxy}: Egy magas rendelkezésre állást biztosító, és
  megbízhatóságot növelő terhelés elosztó eszköz. Konfigurációs fájlokon
  keresztül megszervezhetjük, hogy mely gépet hogyan érjünk el, milyen
  IP címek mely szolgáltatásokhoz tartoznak, illetve round robin módon
  osztja szét a kéréseket az egyes szerverek között. Ez az eszköz csak
  és kizárólak a HTTP TCP kéréseket tudja elosztani, de egyszerű könnyen
  telepíthető, és könnyen kezelhető (ha nem dinamikusan változnak a
  fürtben lévő gépek, mert ha igen akkor szükséges egy mellékes frissítő
  logika is)
\item
  \textbf{ngnix}: Az Nginx egy nyilt forráskódú web kiszolgáló és
  reverse proxy szerver, amivel nagy méretű rendszereket kezelhetünk, és
  segít az alkalmazás biztonságának megörzésében. A kiterjesztett
  változatával (Nginx Plus) képesek lehetünk a terhelés elosztásra, és
  alkalmazás telepítésre. Nem teljesen a proxy szerver szerepét váltja
  ki, de képes elvégezni azt.
\end{itemize}

\subsection{Virtualizálás:}\label{virtualizuxe1luxe1s}

A microservice architektúrák kialakításánál nagy előnyt jelenthet, ha
valamilyen virtualizációt használunk fel a környezet kialakításához.
Virtualizált környezetben könnyebb a telepítés, skálázás, és a
monitorozás is egyszerűbb lehet.

\begin{itemize}
\item
  \textbf{Docker}: Egy konténer virtualizációs eszköz, amelynek
  segítségével egy adott kernel alatt több különböző környezettel
  rendelkező alkalmazásokat futtató környezetet hozhatunk létre. A
  Docker egy szeparált fájlrendszert hoz létre a host gépen, és abban
  hajt végre műveleteket. Készíthetünk vele előre elkészített alkalmazás
  környezeteket, és szolgáltatásokat, ami ideálissá teszi microservice
  architektúrák létrehozásánál. A Docker konténerek segítségével
  egyszerűen telepíthetjük, skálázhatjuk, és fejleszthetjük a rendszert.
\item
  \textbf{libvirt}: Többféle virtualizációs technológiával egyűtt működő
  eszköz, amivel könnyedén irányíthatjuk a virtuális gépeket, és a
  virtualizálás komolyabb részét el absztrahálja. Támogat KVM-em, XEN-t,
  VirtualBox-ot LXC, és sok más virtualizáló eszköt. Ezzel az eszközzel
  a környezet kialakítását szabhatjuk meg, tehát a
  hardware-eserőforrások megosztásában nyújt nagy segítséget.
\item
  \textbf{kvm}: A KVM egy kernel szintű virtualizációs eszköz, amivel
  virtuális gépeket tudunk készíteni. Processzor szintjén képes
  szétválasztani az erőforrásokat, és ezzel szeparált környezeteket
  létrehozni. Virtualizál a processzoron kívül hálózati kártyát,
  háttértárat, grafikus meghajtót, és sok mást. A KVM egy nyilt
  forrűskódú projekt és létrehozhatunk vele Linux és Windows gépeket is
  egyaránt.
\item
  \textbf{Akármilyen cloud}: Ha virtualizációról beszélünk, akkor adja
  magát hogy a CLoud-os környezeteket is ide értsük. Egy microservice
  architektúrájú programot a legcélszerűbb valamilyen Cloud-os
  környezetben létrehozni, mivel egy ilyen környezetnek definiciója
  szerint tartalmaznia kell egy virtualizációs szintet, megosztott
  erőforrásokat, monitorozást, és egyfajta leltárat a futó példányokról.
  Ennek megfelelően a microservice architektúra minden környezeti
  feltételét lefedi, csupán a szolgáltatásokat, business logikát, és az
  interfészeket kell elkészítenünk. Jellemzően a Cloud-os környezetek
  tartalmaznak terhelés elosztást, és skálázási megoldást is, amivel
  szintén erősítik a szolgáltatás alapú architektúrákat. Ilyen környezet
  lehet az Amazon, Microsoft Azure, Google App Engine, OpenStack, és
  sokan mások.
\end{itemize}

\subsection{Service registy:}\label{service-registy}

Számon kell tartani, hogy milyen szolgáltatások elérhetők, milyen címen
és hány példányban az architektúránkban, és ehhez valamilyen
szolgáltatás nyilvántartási eszközt kell használnunk.

\begin{itemize}
\item
  \textbf{Euraka}: Az Eureka a Netflix fejlesztése, egy AWS környezetben
  működő terhelés elosztó alkalmazás, ami figyeli a felvett
  szolgáltatásokat, és így mint nyilvántartás is megfelelő. A
  kommunikációt és a kapcsolatot egy Java nyelven írt szerver és kliens
  biztosítja, ami a teljes logikát megvalósítja. EGyütt működik a
  Netflix álltal fejlesztett Asgard nevezetű alkalmazással ami az AWS
  szolgáltatásokhoz való hozzáférést segíti. Ugyan ez az eszköz erősen
  optimalizált az Amazon Cloud szolgáltatásaihoz, de a leírás alapján
  megállja a helyét önállóan is. Mivel nyilt forráskódú, mintát
  szolgáltat egyéb alkalmazásoknak is.
\item
  \textbf{Consul}: Korábban már említettem ezt az eszközt, mivel abban
  segít, hogy felismerjék egymást a szolgáltatások. A kapcsolatot
  vizsgáló és felderítő logikán kívül tartalmaz egy nyilvántartást is a
  beregisztrált szolgáltatásokról, amiknek az állapotát is
  vizsgálhatjuk.
\item
  \textbf{Apache Zookeeper}: A Zookeeper egy központosított szolgáltatás
  konfigurációs adatok és hálózati adatok karbantartására, ami támogatja
  az elosztott működést, és a szerverek csoportosítását. Az alkalmazást
  elosztott alkalmazás fejlesztésre, és komplex rendszer felügyeletére
  és telepítés segítésére tervezték. A conzulhoz hasonlóan működik, és a
  feladata is ugyan az.
\end{itemize}

\subsection{Monitorozás, loggolás:}\label{monitorozuxe1s-loggoluxe1s}

Ha már megépítettük a microservice architektúrát, akkor meg kell
bizonyosodnunk róla, hogy minden megfelelően működik, és minden rendben
zajlik a szolgáltatásokkal. Ehhez többféle módon és többféle eszközzel
is hozzáférhetünk, mivel az alkalmazás hibákat egy log szerver, a
környezeti problémákat egy monitorozó szerver tudja megfelelően
megmutatni számunkra.

\begin{itemize}
\item
  \textbf{Zabbix}: A Zabbix egy sok területen felhasznált, több 10 ezer
  szervert párhuzamosan megfigyelni képes, akármilyen adatot tárolni
  képes monitorozó alkalmazás, ami képes elosztott működésre, és
  virtuális környezetekben jól használható. Ágens nélküli és ágenses
  adatgyűjtésre is képes, és az adatokat különböző módokon képes
  megjeleníteni (földrajzi elhelyezkedés, gráfos megjelenítés, stb.).
  Nem egészen a microservice architektúrákhoz lett kialakítva, de egy
  elég általános eszköz, hogy felhasználható legyen ilyen célra is.
\item
  \textbf{Kibana + LogStash}: A Kibana egy ingyenes adatmegjelenítő és
  adatfeldolgozó eszköz, amit az elasticsearch fejlesztett ki, és a
  logstash pedig egy log server, amivel tárolhatjuk a loggolási
  adatainkat, és egyszerűen kereshetünk benne. Kifejezetten
  adatfeldolgozásra szolgál mind a két eszköz, és közvetlenül
  együttműködnek az elasticsearch alkalmazással.
\item
  \textbf{Sensu}: A Sensu egy egyszerű monitorozó eszköz, amivel
  megfigyelhetjük a szervereinket. Támogatja Ansible Chef, Puppet
  használatát, és támogatja a Plugin szerű bővíthetőséget. A felülete
  letisztult és elég jó áttekintést ad a szerverek állapotáról. Figyel a
  dinamikus változásokra, és gyorsan lekezeli a változásokkal járó
  riasztásokat. Ezek a tulajdonságai teszik a Cloud-okban könnyen és
  hatékonyan felhasználhatóvá.
\item
  \textbf{Cronitor}: Ez a monitorozó eszköz mikró-szolgáltatások és cron
  job-ok megfigyelésére lett kifejlesztve, HTTP-n keresztül kommunikál,
  és a szologáltatások állapotát figyeli. Nem túl széleskörű eszköz,
  azonban ha csak a szolgáltatások állapota érdekel hasznos lehet, és
  segíthet a Service Registry képzésében is.
\item
  \textbf{Ruxit}: Egy Cloud-osított monitorozó eszköz, amivel
  teljesítmény monitorozást, elérhetőség monitorozást, és figyelmeztetés
  küldést végezhetünk. Az benne a különleges, hogy mesterséges
  intelligencia figyeli a szervereket, és kianalizálja a szerver
  állapotát, és a figyelmeztetéseket is követi. Könnyen skálázható, és
  használat alapú bérezése van. Ez a választás akkor jön jól, ha olyan
  feladatot szánunk az alkalmazásunknak, ami esetleg időben nagyon
  változó terhelést mutat, és az itt kapot riasztások szerint akarunk
  skálázni.
\end{itemize}

\subsection{Hivatkozások:}\label{hivatkozuxe1sok-2}

{[}1{]} Sebastián Peyrott:
\href{https://auth0.com/blog/2015/10/02/an-introduction-to-microservices-part-3-the-service-registry/}{An
Introduction to Microservices, Part 3: The Service Registry}

{[}2{]} Mrina Natarajan:
\href{http://devops.com/2015/05/07/3-golden-rules-microservices-deployments/}{3
golden rules of microservices deployments}

{[}3{]} ElasticBox:
\href{https://elasticbox.com/how-it-works}{ElasticBox}

{[}4{]} ElasticBox: \href{https://elasticbox.com/kubernetes}{Kubernetes}

{[}5{]} Jenkins:
\href{https://wiki.jenkins-ci.org/display/JENKINS/ElasticBox+CI}{ElasticBox
CI}

{[}6{]} Chris Richardson:
\href{https://www.nginx.com/blog/deploying-microservices/}{Choosing a
Microservices Deployment Strategy}

{[}7{]} Zohar Arad:
\href{http://zohararad.github.io/presentations/micro-services-monitoring/}{Effectively
Monitor Your Micro-Service Architectures}

{[}8{]} Nemeth Gergely:
\href{https://www.loggly.com/blog/monitoring-microservices-three-ways-to-overcome-the-biggest-challenges/}{Monitoring
Microservices}

{[}9{]} Cronitor:
\href{https://cronitor.io/help/micro-service-monitoring}{Monitoring
Microservices}

{[}10{]} Ruxit:
\href{https://ruxit.com/microservices/\#microservices_start}{Microservice
monitoring}

{[}11{]} Chris Richardson:
\href{http://microservices.io/patterns/service-registry.html}{Service
registry pattern}

{[}12{]} David Liu:
\href{https://github.com/Netflix/eureka/wiki/Eureka-at-a-glance}{Euraka
at glance}

{[}13{]} Hashicorp: \href{https://www.consul.io/}{Consul}

{[}14{]} Apache: \href{http://zookeeper.apache.org/}{Apache Zookeeper}

{[}15{]} Jenkins: \href{https://jenkins.io/index.html}{Jenkins}

{[}16{]} Netflix: \href{https://github.com/Netflix/eureka/wiki}{Eureka}

{[}17{]} Chef Software Inc.: \href{https://www.chef.io/chef/}{Chef}

{[}18{]} Ansible: \href{https://www.ansible.com/}{Ansible}

{[}19{]} SlatStack inc.: \href{http://saltstack.com/}{SaltStack}

{[}20{]} Amazon:
\href{https://aws.amazon.com/elasticloadbalancing/}{Elastic Load
Balancing}

{[}21{]} HAProxy: \href{http://www.haproxy.org/}{The Reliable, High
Performance TCP/HTTP Load Balancer}

{[}22{]} Fideloper LLC:
\href{https://serversforhackers.com/load-balancing-with-haproxy}{Load
Balancing with HAProxy}

{[}23{]} Nginx: \href{https://www.nginx.com/}{Nginx}

{[}24{]} Docker: \href{https://www.docker.com/}{Docker}

{[}25{]} Libvirt: \href{https://libvirt.org/}{libvirt: The
virtualization API}

{[}26{]} KVM: \href{http://www.linux-kvm.org/page/Main_Page}{Kernel
Virtual Machine}

{[}27{]} Zabbix: \href{http://www.zabbix.com/product.php}{What is
Zabbix}

{[}28{]} Elasticsearch:
\href{https://www.elastic.co/products/kibana}{Kibana}

{[}29{]} Elasticsearch:
\href{https://www.elastic.co/products/logstash}{LogStash}

{[}30{]} Cronitor.io: \href{https://cronitor.io/}{Cronitor}

{[}31{]} dynatrace:
\href{https://ruxit.com/why-ruxit/overview/\#whyruxitoverview_start}{Ruxit
overview}

\chapter{A Markdown-sablon
használata}\label{a-markdown-sablon-hasznuxe1lata}

Ebben a fejezetben röviden, implicit módon bemutatjuk a sablon
használatának módját, ami azt jelenti, hogy sablon használata ennek a
dokumentumnak a forráskódját tanulmányozva válik teljesen világossá.
Amennyiben a szoftver-keretrendszer telepítve van, a sablon alkalmazása
és a dolgozat szerkesztése Markdownban a sablon segítségével
tapasztalataink szerint jóval hatékonyabb, mint egy WYSWYG (\emph{What
You See is What You Get}) típusú szövegszerkesztő esetén (pl. Microsoft
Word, OpenOffice).

A \LaTeX~tördelőrendszer használatához képest a Markdown nyelv több
megszorítást is tartalmaz, így az elkészített dokumentum általában
kevésbé testreszabható. Cserébe viszont a Markdownban készült
dokumentumok exportálhatók HTML vagy e-könyv (EPUB) formátumba is.

\section{Ábrák és táblázatok}\label{uxe1bruxe1k-uxe9s-tuxe1bluxe1zatok}

A képeket a veszteségmentes PNG, valamint a veszteséges JPEG formátumban
érdemes elmenteni.

Az egyes képek mérete általában nem, de sok kép esetén a dokumentum
összmérete így már szignifikáns is lehet. A dokumentumban felhasznált
képfájlokat a dokumentum forrása mellett érdemes tartani, archiválni,
mivel ezek hiányában a dokumentum nem fordul újra. Ha lehet, a
vektorgrafikus képeket vektorgrafikus formátumban is érdemes elmenteni
az újrafelhasználhatóság (az átszerkeszthetőség) érdekében.

A képek beillesztésére a \hyperref[markdown-eszkozok]{Markdown eszközök}
fejezetben mutattunk be \hyperref[markdown-logo]{példát}. Az előző
mondatban egyúttal az automatikusan feloldódó ábrahivatkozásra is
láthatunk példát.

A táblázatok használatára az alábbi táblázat mutat példát.

\begin{longtable}[c]{@{}lll@{}}
\toprule\addlinespace
\begin{minipage}[b]{0.09\columnwidth}\raggedright
Órajel
\end{minipage} & \begin{minipage}[b]{0.14\columnwidth}\raggedright
Frekvencia
\end{minipage} & \begin{minipage}[b]{0.16\columnwidth}\raggedright
Cél pin
\end{minipage}
\\\addlinespace
\midrule\endhead
\begin{minipage}[t]{0.09\columnwidth}\raggedright
CLKA
\end{minipage} & \begin{minipage}[t]{0.14\columnwidth}\raggedright
100 MHz
\end{minipage} & \begin{minipage}[t]{0.16\columnwidth}\raggedright
FPGA CLK0
\end{minipage}
\\\addlinespace
\begin{minipage}[t]{0.09\columnwidth}\raggedright
CLKB
\end{minipage} & \begin{minipage}[t]{0.14\columnwidth}\raggedright
48 MHz
\end{minipage} & \begin{minipage}[t]{0.16\columnwidth}\raggedright
FPGA CLK1
\end{minipage}
\\\addlinespace
\begin{minipage}[t]{0.09\columnwidth}\raggedright
CLKC
\end{minipage} & \begin{minipage}[t]{0.14\columnwidth}\raggedright
20 MHz
\end{minipage} & \begin{minipage}[t]{0.16\columnwidth}\raggedright
Processzor
\end{minipage}
\\\addlinespace
\begin{minipage}[t]{0.09\columnwidth}\raggedright
CLKD
\end{minipage} & \begin{minipage}[t]{0.14\columnwidth}\raggedright
25 MHz
\end{minipage} & \begin{minipage}[t]{0.16\columnwidth}\raggedright
Ethernet chip
\end{minipage}
\\\addlinespace
\begin{minipage}[t]{0.09\columnwidth}\raggedright
CLKE
\end{minipage} & \begin{minipage}[t]{0.14\columnwidth}\raggedright
72 MHz
\end{minipage} & \begin{minipage}[t]{0.16\columnwidth}\raggedright
FPGA CLK2
\end{minipage}
\\\addlinespace
\begin{minipage}[t]{0.09\columnwidth}\raggedright
XBUF
\end{minipage} & \begin{minipage}[t]{0.14\columnwidth}\raggedright
20 MHz
\end{minipage} & \begin{minipage}[t]{0.16\columnwidth}\raggedright
FPGA CLK3
\end{minipage}
\\\addlinespace
\bottomrule
\addlinespace
\caption{Az órajel-generátor chip órajel-kimenetei.}
\end{longtable}

\section{Felsorolások és listák}\label{felsoroluxe1sok-uxe9s-listuxe1k}

Számozatlan felsorolásra mutat példát a jelenlegi bekezdés:

\begin{itemize}
\itemsep1pt\parskip0pt\parsep0pt
\item
  \emph{első bajusz:} ide lehetne írni az első elem kifejését,
\item
  \emph{második bajusz:} ide lehetne írni a második elem kifejését,
\item
  \emph{ez meg egy szakáll:} ide lehetne írni a harmadik elem kifejését.
\end{itemize}

Számozott felsorolást is készíthetünk az alábbi módon:

\begin{enumerate}
\def\labelenumi{\arabic{enumi}.}
\itemsep1pt\parskip0pt\parsep0pt
\item
  \emph{első bajusz:} ide lehetne írni az első elem kifejését, és ez a
  kifejtés így néz ki, ha több sorosra sikeredik,
\item
  \emph{második bajusz:} ide lehetne írni a második elem kifejését,
\item
  \emph{ez meg egy szakáll:} ide lehetne írni a harmadik elem kifejését.
\end{enumerate}

A felsorolásokban sorok végén vessző, az utolsó sor végén pedig pont a
szokásos írásjel. Ez alól kivételt képezhet, ha az egyes elemek több
teljes mondatot tartalmaznak.

Listákban a dolgozat szövegétől elkülönítendő kódrészleteket,
programsorokat, pszeudo-kódokat jeleníthetünk meg.

\begin{verbatim}
* *első bajusz:* ide lehetne írni az első elem kifejését, és ez a kifejtés így néz ki, ha több sorosra sikeredik,
* *második bajusz:* ide lehetne írni a második elem kifejését,
* *ez meg egy szakáll:* ide lehetne írni a harmadik elem kifejését.
\end{verbatim}

A listákban definiálható a használt programozási nyelv (pl.
\texttt{java}, \texttt{latex}, \texttt{bash} stb.).

\section{Képletek}\label{kuxe9pletek}

Ha egy formula nem túlságosan hosszú, és nem akarjuk hivatkozni a
szövegből, mint például a $e^{i\pi}+1=0$ képlet, \emph{szövegközi
képletként} szokás leírni. Csak, hogy másik példát is lássunk, az
$U_i=-d\Phi/dt$ Faraday-törvény a $\mathrm{rot} E=-\frac{dB}{dt}$
differenciális alakban adott Maxwell-egyenlet felületre vett
integráljából vezethető le. Látható, hogy a \LaTeX-fordító a sorközöket
betartja, így a szöveg szedése esztétikus marad szövegközi képletek
használata esetén is.

Képletek esetén az általános konvenció, hogy a kisbetűk skalárt, a kis
félkövér betűk ($\mathbf{v}$) oszlopvektort -- és ennek megfelelően
$\mathbf{v}^T$ sorvektort -- a kapitális félkövér betűk ($\mathbf{V}$)
mátrixot jelölnek. Ha ettől el szeretnénk térni, akkor az alkalmazni
kívánt jelölésmódot célszerű külön alfejezetben definiálni. Ennek
megfelelően, amennyiben $\mathbf{y}$ jelöli a mérések vektorát,
$\mathbf{\vartheta}$ a paraméterek vektorát és
$\hat{\mathbf{y}}=\mathbf{X}\vartheta$ a paraméterekben lineáris
modellt, akkor a *Least-Squares\} értelemben optimális paraméterbecslő
$\hat{\mathbf{\vartheta}}_{LS}=(\mathbf{X}^T\mathbf{X})^{-1}\mathbf{X}^T\mathbf{y}$
lesz.

Emellett kiemelt, sorszámozott képleteket is megadhatunk, ennél az
\texttt{equation} és a \texttt{eqnarray} környezetek helyett a
korszerűbb \texttt{align} környezet alkalmazását javasoljuk (több okból,
különféle problémák elkerülése végett, amelyekre most nem térünk ki).
Tehát

\begin{align}
\dot{\mathbf{x}}&=\mathbf{A}\mathbf{x}+\mathbf{B}\mathbf{u},\\
\mathbf{y}&=\mathbf{C}\mathbf{x},
\end{align}

ahol $\mathbf{x}$ az állapotvektor, $\mathbf{y}$ a mérések vektora és
$\mathbf{A}$, $\mathbf{B}$ és $\mathbf{C}$ a rendszert leíró
paramétermátrixok. Figyeljük meg, hogy a két egyenletben az
egyenlőségjelek egymáshoz igazítva jelennek meg, mivel a mindkettőt az
\& karakter előzi meg a kódban. Lehetőség van számozatlan kiemelt képlet
használatára is, például

\begin{align}
\dot{\mathbf{x}}&=\mathbf{A}\mathbf{x}+\mathbf{B}\mathbf{u},\nonumber\\
\mathbf{y}&=\mathbf{C}\mathbf{x}\nonumber.
\end{align}

Mátrixok felírására az $\mathbf{A}\mathbf{x}=\mathbf{b}$ inhomogén
lineáris egyenlet részletes kifejtésével mutatunk példát:

\begin{align}
\begin{bmatrix}
a_{11} & a_{12} & \dots & a_{1n}\\
a_{21} & a_{22} & \dots & a_{2n}\\
\vdots & \vdots & \ddots & \vdots\\
a_{m1} & a_{m2} & \dots & a_{mn}
\end{bmatrix}
\begin{pmatrix}x_1\\x_2\\\vdots\\x_n\end{pmatrix}=
\begin{pmatrix}b_1\\b_2\\\vdots\\b_m\end{pmatrix}.
\end{align}

A \texttt{\textbackslash{}frac} utasítás hatékonyságát egy általános
másodfokú tag átviteli függvényén keresztül mutatjuk be, azaz

\begin{align}
W(s)=\frac{A}{1+2T\xi s+s^2T^2}.
\end{align}

A matematikai mód minden szimbólumának és képességének a bemutatására
természetesen itt nincs lehetőség, de gyors referenciaként hatékonyan
használhatók a következő linkek:

\begin{itemize}
\itemsep1pt\parskip0pt\parsep0pt
\item
  \url{http://www.artofproblemsolving.com/LaTeX/AoPS_L_GuideSym.php},
\item
  \url{http://www.ctan.org/tex-archive/info/symbols/comprehensive/symbols-a4.pdf},
\item
  \url{ftp://ftp.ams.org/pub/tex/doc/amsmath/short-math-guide.pdf}.
\end{itemize}

Ez pedig itt egy magyarázat, hogy miért érdemes \texttt{align}
környezetet használni:
\url{http://texblog.net/latex-archive/maths/eqnarray-align-environment/}.

\section{Irodalmi hivatkozások}\label{irodalmi-hivatkozuxe1sok}

Az irodalmi hivatkozások alkalmazására javasolt a BiB\TeX~használata,
ezért ez a sablon is ezt támogatja. Ebben az esetben egy külön szöveges
adatbázisban definiáljuk a forrásmunkákat, és egy külön stílusfájl
határozza meg az irodalomjegyzék kinézetét. Ez, összhangban azzal, hogy
külön formátumkonvenció határozza meg a folyóirat-, a könyv-, a
konferenciacikk stb. hivatkozások kinézetét az irodalomjegyzékben (a
sablon használata esetén ezzel nem is kell foglalkoznia a hallgatónak,
de az eredményt célszerű ellenőrizni). A felhasznált hivatkozások
adatbázisa egy \texttt{.bib} kiterjesztésű szöveges fájl, amelynek
szerkezetét az alábbi kódrészlet demonstrálja. A forrásmunkák
bevitelekor a sor végi vesszők külön figyelmet igényelnek, mert hiányuk
a BiB\TeX-fordító hibaüzenetét eredményezi. A forrásmunkákat típus
szerinti kulcsszó vezeti be (\texttt{@book} könyv,
\texttt{@inproceedings} konferenciakiadványban megjelent cikk,
\texttt{@article} folyóiratban megjelent cikk, \texttt{@techreport}
valamelyik egyetem gondozásában megjelent műszaki tanulmány,
\texttt{@manual} műszaki dokumentáció esetén stb.). Nemcsak a megjelenés
stílusa, de a kötelezően megadandó mezők is típusról-típusra változnak.
Egy jól használható referencia a
\url{http://en.wikipedia.org/wiki/BibTeX} oldalon található.

\begin{Shaded}
\begin{Highlighting}[]
\KeywordTok{@BOOK}\NormalTok{\{}\OtherTok{Wettl04}\NormalTok{,}
  \DataTypeTok{author} \NormalTok{= "}\StringTok{Ferenc Wettl and Gyula Mayer and Péter Szabó}\NormalTok{",}
  \DataTypeTok{title} \NormalTok{= "}\CharTok{\textbackslash{}LaTeX}\StringTok{~kézikönyv}\NormalTok{",}
  \DataTypeTok{publisher} \NormalTok{= "}\StringTok{Panem Könyvkiadó}\NormalTok{",}
  \DataTypeTok{year} \NormalTok{= 2004}
\NormalTok{\}}
\KeywordTok{@ARTICLE}\NormalTok{\{}\OtherTok{Candy86}\NormalTok{,}
  \DataTypeTok{author} \NormalTok{= "}\StringTok{James C. Candy}\NormalTok{",}
  \DataTypeTok{title} \NormalTok{= "}\StringTok{Decimation for Sigma Delta Modulation}\NormalTok{",}
  \DataTypeTok{journal} \NormalTok{=  "}\StringTok{\{IEEE\} Trans.\textbackslash{} on Communications}\NormalTok{",}
  \DataTypeTok{volume} \NormalTok{= 34,}
  \DataTypeTok{number} \NormalTok{= 1,}
  \DataTypeTok{pages} \NormalTok{= "}\StringTok{72--76}\NormalTok{",}
  \DataTypeTok{month} \NormalTok{= }\StringTok{jan}\NormalTok{,}
  \DataTypeTok{year} \NormalTok{= 1986,}
\NormalTok{\}}
\KeywordTok{@INPROCEEDINGS}\NormalTok{\{}\OtherTok{Lee87}\NormalTok{,}
  \DataTypeTok{author} \NormalTok{= "}\StringTok{Wai L. Lee and Charles G. Sodini}\NormalTok{",}
  \DataTypeTok{title} \NormalTok{= "}\StringTok{A Topology for Higher Order Interpolative Coders}\NormalTok{",}
  \DataTypeTok{booktitle} \NormalTok{= "}\StringTok{Proc.\textbackslash{} of the IEEE International Symposium on Circuits and Systems}\NormalTok{",}
  \DataTypeTok{year} \NormalTok{= 1987,}
  \DataTypeTok{vol} \NormalTok{= 2,}
  \DataTypeTok{month} \NormalTok{= }\StringTok{may} \NormalTok{# "}\StringTok{~4--7}\NormalTok{",}
  \DataTypeTok{address} \NormalTok{= "}\StringTok{Philadelphia, PA, USA}\NormalTok{",}
  \DataTypeTok{pages} \NormalTok{= "}\StringTok{459--462}\NormalTok{"}
\NormalTok{\}}
\KeywordTok{@PHDTHESIS}\NormalTok{\{}\OtherTok{KissPhD}\NormalTok{,}
  \DataTypeTok{author} \NormalTok{= "}\StringTok{Peter Kiss}\NormalTok{",}
  \DataTypeTok{title} \NormalTok{= "}\StringTok{Adaptive Digital Compensation of Analog Circuit Imperfections for Cascaded Delta-Sigma Analog-to-Digital Converters}\NormalTok{",}
  \DataTypeTok{school} \NormalTok{= "}\StringTok{Technical University of Timi}\CharTok{\textbackslash{}c}\StringTok{\{s\}oara, Romania}\NormalTok{",}
  \DataTypeTok{month} \NormalTok{= }\StringTok{apr}\NormalTok{,}
  \DataTypeTok{year} \NormalTok{= 2000}
\NormalTok{\}}
\KeywordTok{@MANUAL}\NormalTok{\{}\OtherTok{Schreier00}\NormalTok{,}
  \DataTypeTok{author} \NormalTok{= "}\StringTok{Richard Schreier}\NormalTok{",}
  \DataTypeTok{title} \NormalTok{= "}\StringTok{The Delta-Sigma Toolbox v5.2}\NormalTok{",}
  \DataTypeTok{organization} \NormalTok{= "}\StringTok{Oregon State University}\NormalTok{",}
  \DataTypeTok{year} \NormalTok{= 2000,}
  \DataTypeTok{month} \NormalTok{= }\StringTok{jan}\NormalTok{,}
  \DataTypeTok{note} \NormalTok{= "}\CharTok{\textbackslash{}newline}\StringTok{ URL: http://www.mathworks.com/matlabcentral/fileexchange/}\NormalTok{"}
\NormalTok{\}}
\KeywordTok{@MISC}\NormalTok{\{}\OtherTok{DipPortal}\NormalTok{,}
    \DataTypeTok{author} \NormalTok{= "}\StringTok{Budapesti \{M\}űszaki és \{G\}azdaságtudományi \{E\}gyetem, \{V\}illamosmérnöki és \{I\}nformatikai \{K\}ar}\NormalTok{",}
  \DataTypeTok{title} \NormalTok{= "}\StringTok{\{D\}iplomaterv portál (2011 február 26.)}\NormalTok{",}
  \DataTypeTok{howpublished} \NormalTok{= "}\CharTok{\textbackslash{}url}\StringTok{\{http://diplomaterv.vik.bme.hu/\}}\NormalTok{",}
\NormalTok{\}}
\end{Highlighting}
\end{Shaded}

A stílusfájl egy \texttt{.sty} kiterjesztésű fájl, de ezzel lényegében
nem kell foglalkozni, mert vannak beépített stílusok, amelyek jól
használhatók. Ez a sablon a BiB\TeX-et használja, a hozzá tartozó
adatbázisfájl a \texttt{mybib.bib} fájl. Megfigyelhető, hogy az
irodalomjegyzéket a dokumentum végére (a
\texttt{\textbackslash{}end\{document\}} utasítás elé) beillesztett
\texttt{\textbackslash{}bibliography\{mybib\}} utasítással hozhatjuk
létre, a stílusát pedig ugyanitt a
\texttt{\textbackslash{}bibliographystyle\{plain\}} utasítással adhatjuk
meg. Ebben az esetben a \texttt{plain} előre definiált stílust
használjuk (a sablonban is ezt állítottuk be). A \texttt{plain} stíluson
kívül természetesen számtalan más előre definiált stílus is létezik.
Mivel a \texttt{.bib} adatbázisban ezeket megadtuk, a BiB\TeX-fordító is
meg tudja különböztetni a szerzőt a címtől és a kiadótól, és ez alapján
automatikusan generálódik az irodalomjegyzék a stílusfájl által
meghatározott stílusban.

Az egyes forrásmunkákra a szövegből továbbra is a \texttt{@{[}...{]}}
paranccsal tudunk hivatkozni, így a fenti kódrészlet esetén a
hivatkozások rendre \texttt{{[}@Wettl04{]}}, \texttt{{[}@Candy86{]}},
\texttt{{[}@Lee87{]}}, \texttt{{[}@KissPhD{]}},
\texttt{{[}@Schreirer00{]}} és \texttt{{[}@DipPortal{]}}. Az
irodalomjegyzékben alapértelmezésben csak azok a forrásmunkák jelennek
meg, amelyekre található hivatkozás a szövegben, és ez így alapvetően
helyes is, hiszen olyan forrásmunkákat nem illik az irodalomjegyzékbe
írni, amelyekre nincs hivatkozás.

Mivel a fordítási folyamat során több lépésben oldódnak fel a
szimbólumok, ezért gyakran többször (TeXLive és TeXnicCenter esetén
2-3-szor) is le kell fordítani a dokumentumot. Ilyenkor ez első 1-2
fordítás esetleg szimbólum-feloldásra vonatkozó figyelmeztető üzenettel
zárul. Ha hibaüzenettel zárul bármelyik fordítás, akkor nincs értelme
megismételni, hanem a hibát kell megkeresni. A \texttt{.bib} fájl
megváltoztatáskor sokszor nincs hatása a változtatásnak azonnal, mivel
nem mindig fut újra a BibTeX fordító. Ezért célszerű a változtatás után
azt manuálisan is lefuttatni (TeXnicCenter esetén
\texttt{Build/BibTeX}).

Hogy a szövegbe ágyazott hivatkozások kinézetét demonstráljuk, itt most
sorban meghivatkozzuk a \citep{Wettl04}, \citep{Candy86}, \citep{Lee87},
\citep{KissPhD} és az \citep{Schreier00} forrásmunkát, valamint az
\citep{DipPortal} weboldalt.

Megjegyzendő, hogy az ékezetes magyar betűket is tartalmazó
\texttt{.bib} fájl az \texttt{inputenc} csomaggal betöltött
\texttt{latin2} betűkészlet miatt fordítható. Ugyanez a \texttt{.bib}
fájl hibaüzenettel fordul egy olyan dokumentumban, ami nem tartalmazza a
\texttt{\textbackslash{}usepackage{[}latin2{]}\{inputenc\}} sort.
Speciális igény esetén az irodalmi adatbázis általánosabb érvényűvé
tehető, ha az ékezetes betűket speciális latex karakterekkel
helyettesítjük a \texttt{.bib} fájlban, pl. á helyett
\texttt{\textbackslash{}'\{a\}}-t vagy ő helyett
\texttt{\textbackslash{}H\{o\}}-t írunk.

\section{A dolgozat szerkezete és a
forrásfájlok}\label{a-dolgozat-szerkezete-uxe9s-a-forruxe1sfuxe1jlok}

A diplomatervsablon (a kari irányelvek szerint) az alábbi fő
fejezetekből áll:

\begin{itemize}
\itemsep1pt\parskip0pt\parsep0pt
\item
  \emph{tájékoztató} a szakdolgozat/diplomaterv szerkezetéről, ami a
  végső dolgozatból törlendő, valamint a \emph{feladatkiírás}
  (\texttt{guideline.md}), a dolgozat nyomtatott verzójában ennek a
  helyére kerül a tanszék által kiadott, a tanszékvezető által aláírt
  feladatkiírás, a dolgozat elektronikus verziójába pedig a
  feladatkiírás egyáltalán ne kerüljön bele, azt külön tölti fel a
  tanszék a diplomaterv-honlapra,
\item
  \emph{címoldal} (generált),
\item
  \emph{tartalomjegyzék} (generált),
\item
  a diplomatervező \emph{nyilatkozat}a az önálló munkáról (generált),
\item
  1-2 oldalas tartalmi \emph{összefoglaló} magyarul és angolul, illetve
  elkészíthető még további nyelveken is (\texttt{abstract.md}),
\item
  \emph{fejezetek}: \emph{bevezetés} (feladat értelmezése, a tervezés
  célja, a feladat indokoltsága, a diplomaterv felépítésének rövid
  összefoglalása, \texttt{chapter1.md}), a feladatkiírás pontosítása és
  részletes elemzése, előzmények (irodalomkutatás, hasonló alkotások),
  az ezekből levonható következtetések, a tervezés részletes leírása, a
  döntési lehetőségek értékelése és a választott megoldások indoklása, a
  megtervezett műszaki alkotás értékelése, kritikai elemzése,
  továbbfejlesztési lehetőségek (\texttt{chapter\{2..n\}.md}),
\item
  összefoglalás (\texttt{summary.md}),
\item
  esetleges \emph{köszönetnyilvánítás}ok (\texttt{acknowledgement.md}),
\item
  részletes és pontos \emph{irodalomjegyzék} (generált),
\item
  \emph{függelékek} (\texttt{appendices.md}).
\end{itemize}

\chapter{Összefoglaló}\label{uxf6sszefoglaluxf3}

A diplomaterv összefoglaló fejezete.

\chapter*{Köszönetnyilvánítás}\label{kuxf6szuxf6netnyilvuxe1nuxedtuxe1s}
\addcontentsline{toc}{chapter}{Köszönetnyilvánítás}

A köszönetnyilvánítás nem kötelező, akár törölhető is. Ha a szerző
szükségét érzi, itt lehet köszönetet nyilvánítani azoknak, akik
hozzájárultak munkájukkal ahhoz, hogy a hallgató a szakdolgozatban vagy
diplomamunkában leírt feladatokat sikeresen elvégezze. A konzulensnek
való köszönetnyilvánítás sem kötelező, a konzulensnek hivatalosan is
dolga, hogy a hallgatót konzultálja.

\listoftables
\listoffigures

\bibliography{bibliography}


\appendix

\chapter{Függelék}\label{fuxfcggeluxe9k}

\section{Válasz az ,,Élet, a világmindenség, meg minden''
kérdésére}\label{vuxe1lasz-az-uxe9let-a-viluxe1gmindensuxe9g-meg-minden-kuxe9rduxe9suxe9re}

A Pitagorasz-tételből levezetve

\[c^2=a^2+b^2=42.\]

A Faraday-indukciós törvényből levezetve

\[\mathrm{rot} E=-\frac{dB}{dt} \quad \longrightarrow \quad U_i=\oint\limits_\mathbf{L}{\mathbf{E}\mathbf{dl}}=-\frac{d}{dt}\int\limits_A{\mathbf{B}\mathbf{da}}=42.\]

\end{document}
